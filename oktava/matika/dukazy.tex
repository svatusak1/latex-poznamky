\documentclass{article}

\usepackage{amsfonts} 
\usepackage{amsmath} 
\usepackage{amsthm}
\usepackage[most]{tcolorbox}
\usepackage[shortlabels]{enumitem}
\usepackage{txfonts}
\usepackage{mathtools}  

\mathtoolsset{showonlyrefs}  
\theoremstyle{definition}
\newtheorem{exmp}{Example}[section]

\begin{document}
\section{Důkazy}


matematicke vety $\Rightarrow a \Leftrightarrow$
Ekvivalence implikace a implikace

Typy dukazu

\subsection{Primy}
chci $a \Rightarrow B$
$A \Rightarrow A_1 \Rightarrow A_2 \dots \Rightarrow A_n \Rightarrow B$

\begin{exmp}
    $(2|x \land 3|x \Rightarrow 6|x)$
    \begin{gather}
        2|x \Rightarrow \exists k \in \varmathbb{N}: x = 2k \Rightarrow  3|x \\
        \Rightarrow \exists l \in \varmathbb{N}: x = 3l \Rightarrow 2k = 3k \Rightarrow 3|k \\ \Rightarrow 
        \exists m \in \varmathbb{N}: k = 3m \Rightarrow x = 2k = 2 \cdot 3 \cdot m \Rightarrow x = 6m \Rightarrow 6|x
    \end{gather}
\end{exmp}
\subsection{Neprimy} misto $A \Rightarrow B$ dokazujeme $B' \Rightarrow A'$
\subsection{Sporem} dokazeme, ze negace neplati $\Rightarrow$ vyrok plati \\
zkousim dokazat negaci $(A\land B') \Rightarrow \dots \Rightarrow $ spor \\
spor = zjevne nepravdive tvrzeni 

\begin{itemize}
\item popreni predpokladu
\item obecny nesmysl
\end{itemize}

\subsection{Dukazy prirozenych cisel}
\subsubsection{Matematicka indukce}
$ \forall n \in \varmathbb{N}: V(n)$
\begin{enumerate}
    \item dokazeme pro $n = 1 \dots V(1)$ plati
    \item dokazujeme, ze $V(n) \Rightarrow V(n+1)$
\end{enumerate}

\subsubsection{Dukaz existencnim tvrzenim}
$\exists x: \dots$ staci jedno x najit / sestrojit a hotovo
\subsubsection{Vyvrácení}
$\forall x \dots$ najdeme protipriklad

\begin{exmp}

    dokazte, ze $\forall n \in \varmathbb{N}: 2|(n^2+m)$
    $n^2 +n = n (n+1)$
    rozdelime na suda a licha
    \begin{enumerate}

        \item $n = 2k$
            $2|n \Rightarrow 2k(n+1)$

        \item $n = 2k+1$
            $n(2k+1+1) = n(2(k+1))=2n(k+1)$

    \end{enumerate}

\end{exmp}

\begin{exmp}

    vyslovte hypotezu o nejvetsim spolecnem deliteli vyrazu $n^4 - n^2$ a dokazte ji
    \begin{center}
\begin{tabular}{c | c}
    n & $n^4 - n^2$ \\
    \hline
    0 & 0 \\
    1 & 0 \\
    2 & 12 \\
    3 & 72 \\
    4 & 240
\end{tabular}
    \end{center}
$\forall n \in \varmathbb{N}: 12 | n^4-n^2$
\[n^4-n^2 = n^2(n^2-1)=n^2(n\pm1) = (n-1)n^2\dots(n+1) \]
\[12 | n \Rightarrow 3|n \land 4|n \]
\begin{enumerate}[a)]

    \item $ 3|n $
        \begin{itemize}
            \item $n = 3k$
                $3|n^2$
            \item $n = 3k+1$
                $3|(n-1)$
            \item $n= 3k+2$
                $3|(n+1)$
    \end{itemize}
\item $4|n$
    \begin {itemize}
\item $n = 2k $
$4|n^2$
\item $n = 2k+1 $
    $4|(n-1)(n+1)$
\end{itemize}


\end{enumerate}
$\Box$

\end{exmp}

\begin{exmp}

    kdyz je ciferny soucet delitelny, pak je n delitelne 3
    $n = a_n * 10^n + a_n-1 * 10^n-1 + \dots + a_1*10 + a_0$
    $s = a_n +a_n-1 +\dots+ a_1 +a_0$
    $n = a_n (1 + (10^n-1))+ \dots + a_1(1+9)+1_0 = (a_n + a_n-a \dots + a_1 +a_0) + a_n (10^n-1) + \dots 99a_2 + 9 a_1$ 
    3 i 9 deli druhou cast a ze zadani vime, ze i prvni
    soucet dvou delitelu trema je taky delitelny trema
    cislo minus jeho ciferny soucet je delitelne trema

\end{exmp}


\begin{exmp}

    Dokazte sporem, ze $ \sqrt 3 \not \in \varmathbb{Q}$
    predpokladame negaci, tj.
    $\sqrt 3 = \frac p q  | \cdot p$ p q nesoudelna
    $\sqrt 3 \cdot q = p  | ^2$
    $3 q^2 = p^2 \Rightarrow 3|p^2 \Rightarrow 3|p \Rightarrow p = 3k$
    $3q^2 = (3k)^2 = 9k^2 | :3$
    $q^2 = 3k^2 \Rightarrow 3|q^2 \Rightarrow 3|q$
    spor s predpokladem

\end{exmp}

\begin{exmp}

    Dokazte, ze prvocisel je nekonecne 
    predpokladam konecny pocet
    $P = {2, 3, 5, 7, \dots, p_{n-1}, p_n}$
    $x = {velky pi}^n_{i = 1} p_i$
    $y = x+1$
    y neni delitelne zadnym prvocislem a pritom $y > p_n$
    spor $y \not \in P$
    
\end{exmp}

\begin{exmp}

    $\forall n \in \varmathbb{N}: 3| 2^{2n}-7$
    MI
    \begin{enumerate}
        \item $n=1 $
            $2^2n-7 = -3 \dots 3|-3$
        \item $V(n) \Rightarrow V(n+1)$
            $2^{2(n+1)} -7 = 2^{2n+2} - 7 = 4 \cdot 2 ^{2n} - 7$
            $=(3+1)2^{2n}-7 = 3 \cdot 2^{2n} + 2^{2n} - 7$
            druha cast delitelna 3 podle predpokladu; 3|prvni cast
            Q.E.D.
    \end{enumerate}

\end{exmp}

\begin{exmp}

    \[\forall n \in \varmathbb{N}: 100|\sum_{i=1}^{4n} 7^i\]
    \begin{enumerate}
        \item n = 1 \dots $100|2800$
        \item $n+1 \dots 7^4n+ + 7^{4n+1} + 7^{4n+2} + 7^{4n+3} + 7^{4n+4} = 7^{4n} (7 + 7^2+ 7^3 + 7^4) = 7^{4n} +2800$
    \end{enumerate}

\end{exmp}

\begin{exmp}
    MI
$2|5n^2-n \Leftrightarrow 2|n(5n-1)$

    \begin{enumerate}
        \item $ n = 1 \Rightarrow 2|4$ pravda
        \item $V(n) \Rightarrow V(n+1)$
    \end{enumerate}
\end{exmp}


    $\forall n \in \varmathbb{N}: (1+2+3+\dots+n)^2 = 1^3+2^3+3^3+\dots+n^3$


\end{document}
