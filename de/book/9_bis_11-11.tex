\documentclass{scrbook}

\usepackage{amsfonts}
\usepackage{amsmath}
\usepackage{mathtools}
\usepackage{wrapfig}
\usepackage{marvosym}
\let\marvosymLightning\Lightning
\usepackage{pgfplots}
\pgfplotsset{width=10cm,compat=1.9}

\usepackage{multicol}
\usepackage{xcolor}
\usepackage{soul}
\newcommand{\mathcolorbox}[2]{\colorbox{#1}{$\displaystyle #2$}}

\newtheorem{definition}{Definition}
\begin{document}

\chapter{Ableitung}

\section{Bestimmung der Ableitungsfunktion mit dem Quotient}
\(
gegeben: f_{(x)} = 4 - \frac 1 4 x^2 \qquad D_f = \mathbb{R}  \\
\newline
gesucht: f'_{(x)}
\)

\[m_{(h)} = \frac{f_{a+h}-f_{(a)}}h = \frac{4-\frac 1 4 (a+h)^2-(4-\frac 1 4 a^2)}h \]
\[= \frac{4-\frac 1 4 (a^2+2 a h + h^2) - 4 + \frac 1 4 a^2}h \]

\[= \frac{4-\frac {a^2} {4} - \frac {ah} {2} - \frac {h^2} {4} - 4 + \frac {a^2} 4}h \]

\[= \frac {-\frac {a h} {2} - \frac {h^2} 4}h = -\frac a 2 - \frac h 4\]

\[f'_{(a)} = \lim_{h\to0} m_{(h)} = \lim_{h\to0}-\frac a 2 -\frac h 4 = -\frac a 2\]

\[f'_{(x)} = -\frac 1 2 x \]

\section{Ableitungs Regel und höhere Ableitungen}
\begin{definition}
Ist f' differenzierbar, so enthält man durch Ableiten von f' der zweite Ableitung f'' usw.
\[f''(x) = [f'(x)]'\]
\[f^{(x)}(x) = [f^{(n-1)}(x)]'\]
\end{definition}

\begin{definition}
Polynom von Grad n:
\[ a_n \neq0\quad a_n, \dots, a_o \in \mathbb{R}\]
\[a_n x^n+a_{n-a}x^{n-1}+\dots+a_1 x^1+a_0\]
\end{definition}
\begin{definition}
Für verschachtelte Funktionen gilt:
\[f(x) = u(v(x)) \quad f = u\circ v:x\quad f(x) = (u \circ v)(x) \quad \circ \text{ist nicht komutativ}\]
\end{definition}

\begin{definition}Summeregel: 
\[f(x) = g(x) + v(x)\]
\[ \Rightarrow f'(x) = g'(x) + v'(x)\]
\end{definition}

\begin{definition}Potenzregel: 
\[f(x) = x^r \text{mit r} \in \mathbb{R} \text{\textbackslash} \{0\}\]
\[ \Rightarrow f'(x) = r \cdot x^{r-1}\]
\end{definition}

\begin{definition}Faktorregel:
\[f(x) = c \cdot g(x) \text{mit c} \in \mathbb{R}\]
\[ \Rightarrow f'(x) = c \cdot g'(x)\]
\end{definition}

\begin{definition} Kettenregel: 
\[ f(x) = u(v(x))\]
\[ \Rightarrow f'(x) = u'(v(x))\cdot v'(x) \quad f' = (u' \circ v)\cdot v'\]
\end{definition}

\begin {definition} Produktregel:
\[f(x) = g(x)\cdot v(x)\]
\[\Rightarrow f'(x) = g'(x)\cdot v(x) + v'(x)\cdot g(x)\]
\end{definition}

\chapter{Monotonie}

\begin{definition}
\[ \forall x_1, x_2 \in I_1: x_1 < x_2 \land f(x_1) < f(x_2) \implies f(x) \text{ ist streng monoton wachsend in }I_1\]
\[ \forall x_1, x_2 \in I_1: x_1 < x_2 \land f(x_1) > f(x_2) \implies f(x) \text{ ist streng monoton fallend in }I_1\]
\end{definition}

\subsection{Rechnerische Bestimmung der Monotomie Intervalle mit dem Monotoniesatz}
\begin{enumerate}
\item Bestimmung der Nullstellen der 1. Ableitung
\item Definition der Intervalle
\item Werte von Interval und f' entesetzen
\end{enumerate}

\subsubsection{Beispiel}
\[f(x) = \frac 13 x^3 - 3x^2 +3x+1\]
\[f'(x) = x^2 - 6x + 3\]
\begin{enumerate}
\item \[f'(x) = 0 \Rightarrow x^2 - 6x + 3 = 0\]
\[x = 3\pm \sqrt6\]
\item 
\begin{itemize}
\item $I_1 = ]-\infty;3-\sqrt6[ $
\item $I_2 = ]3-\sqrt6; 3+\sqrt6[$
\item $I_3 = ]3+\sqrt6; \infty[$
\end{itemize}
\item 
\begin{itemize}
\item $f'(-10) \in \mathbb{R}^+ \implies f(x) \in I_1$ ist s.t. wachsend
\item $f'(0) \in \mathbb{R}^- \implies f(x) \in I_1$ ist s.t. fallend
\item $f'(10) \in \mathbb{R}^+ \implies f(x) \in I_1$ ist s.t. wachsend
\end{itemize}
\end{enumerate}

\chapter{Tangenten und Normalen Gleichungen}

\begin{tikzpicture}
\begin{axis}
\addplot[color=black]{(x-6)*(x-5)*(x-7) + 4};
\end{axis}
\end{tikzpicture}

\[t: y = m_t x + c\]
\[m_t = f'(u) \quad x = u \quad y = f(u)\]
\[f(u) = f'(u) \cdot u+c\]
\[ c = f(u) - f'(u) \cdot u \]
\[y_t = f'(u) x + f(u) - f'(u)\cdot u \]
\[\mathcolorbox {red} {y_t = f(u)(x-u) + f(u)}\]
\[t \perp n \Leftrightarrow m_n \cdot m_t = -1\; ; \; m_n = \frac{-1}{m_t}\]
\[\mathcolorbox {cyan} {y_n = -\frac1{f'(u)}(x-u)+f(u)}\]

\subsection{Bestimmung dei Tangentegliechung durch eine aüßere Punkt}
geg: $f(x) = \frac12 x^3 \; ; \; C = [0; -1]$

\[f(x) = \frac32 x^2\]
\[y_t = f'(u)(x-u) + f(u)\]
\center{C einsetzen}
\[-1 = \frac32 u^2(0-u)+\frac12 u^3\]
\[-1 = -\frac32 u^3+\frac12 u^3\]
\[-1 = -u^3\]
\[u = 1 \Rightarrow \text{Berührpunkt} = B[1;f(1)]\]

\begin{equation*}
\begin{aligned}
t: y &= f'(u)(x-u)+f(u)
\\
&=\frac32 1^2(x-1)+\frac12 1^3
\\
&=\frac32 x - \frac 32 + \frac12
\\
&=\frac32 x -1
\\
n: y &= -\frac1{f'(u)}(x-u)+f(u)
\\
&=-\frac23 1^2(x-1) +\frac12 1^3
\\
& = -\frac23 x +\frac23 + \frac12
\\
&= -\frac23 x + \frac 76
\end{aligned}
\end{equation*}   



\chapter{Krummung}

\section{Wendestellen}

\begin{definition}
Gegeben ist eine auff einem Intervall I definierte Funktion f, die dort zweimal differenzierbar ist. Ist $x_0$ eine inner Stelle von I, so gilt:

\end{definition}

\chapter{Nebenbedingungen}

Anwendungssituationen in denen eine Größe von zwei Variablen, die miteinander einen Zusammenhang haben, abhängt.

\section{Methoden für lösen die Nebnbedingungenprobleme}
\begin{multicols}{2}

\begin{enumerate}
\item Formel für die Größe aufstellen, die extremal werden soll.
\item Zusamenhang zwischen den beiden Variblen beschreiben (\textbf{Nebenbedingungen})
\item \textbf{Zielfunktion} bestimmen durch Einsetzen in die Ausgangsformel
\item \textbf{Definitionsmenge} festlegen
\item Untersuchen der Zielfunktion auf \textbf{Extremwerte}
\item Ränder der Definitionsmenge untersuchen
\item Ergebnis formulieren
\end{enumerate}

\columnbreak
\begin{enumerate}
\item $A= u \cdot v$
\\
$v = f(u) = -0,8u + 4,8$
\item v hängt von u ab
\item $A(u) = u \cdot f(u) = -0,8u^2 + 4,8u$
\item $u \in [0;6]$
\item $A'(u) = 0 \rightarrow -0,8u^2 + 4,8u = 0$ liefert u = 3
\\
$A''(u) = -1,6 < 0$ für alle u
\\
Stelle des lokales Maximum: $u = 3$
\item $A(0) = 0$ und $A(6) = 0$
\item Für $u = 3$ wird der Flächeinhalt des Rechtecks am größten: $A(3) = 7,2$
\end{enumerate}

\end{multicols}

\subsection{Abstandformel}
Abstand zweir Punkte $P(x_p\vert y_p)$ und $Q(x_q\vert y_q)$ im Koordinatensystem
\[d = d(Q, P) = \overline{QP} = \sqrt{(x_p - x_q)^2 + (y_p - y_q)^2}\]
\chapter{Exponential und Logarithmus funktionen}

\begin{definition}
Eine Funktion $f$ mit $f_{(x)} = c \cdot b^x,\, b> 0, \,b \neq 1, \,c \in \mathbb{R}$ heißt \textbf{Exponentialfunktion mit Basis b.}
\end{definition}

\begin{definition}
Ist $ b^x = y$ mit $y>0$ und $b>0$ sowie $ b \not = 1$, so nennt man den Exponent x den \b Logarithmus von y zur Basis b.\b
Man schreibt: $x = log_b(y)$
\end{definition}

\section{Euler-Zahle}

\subsection{Die natürliche Exponentialfunktion und ihre Ableitung, die Euler-Zahle}

\underline{Vermutung: }
\begin{itemize}
\item Für Exponentielfunktion mit $f_{(x)} = b^x (b > 0, b \neq 1)$ gilt: 
\end{itemize}
\[\frac{f'_{(x)}}{f_{(x)}} = k \quad (k \in \mathbb{R}) \Rightarrow \]
\[f'_{(x)} = k \cdot f_{(x)}\]
\[f'_{(x)} \approx f_{(x)}\]


\begin{definition}
Sei e $\approx 2, 7182818284$ die Eulersche Zahle, dann heißt die Funtion mit $f_{(x)} = e^x$ \textbf{natürliche Exponentialfunktion}
\end{definition}

Zusatz: Die Ableitung stimmt mit der Funktion überein $f'_{(x)} = e^x$


\begin{definition}
Für alle $ y > 0$ gilt: $e^{ln(y)} = y$
Für alle $ x \in \mathbb{R} $ gilt: $ ln (e^x) = x$
\end{definition}

Der Logarithmus zur Basis e heißt \textbf{ natürliches Logarithmus}
\[log_e = ln\]
Bemerkung: $ ln(e) = 1\; ; \; ln(1) = 0$

\begin{definition}

Die Zahl $ x = ln (b)$ löst die \textbf{Exponentialgleichung $ e^x = b \;(b > 0)$}
\end{definition}
\[ \Rightarrow e^x = 5 \Rightarrow x = ln (5) \Rightarrow 5 = ln (e^5)\]


\subsection{Die Ableitung von $f_{(x)} = e^x$}
\underline{Beispiel}:

\[f_{(x)} = 3 \cdot e ^{5x + 1} = u(v(x)) \]
\[u_{(x)} = 3e^x \rightarrow u'_{(x)} = 3e^x\]
\[v_{(x)} = 5x+1 \rightarrow v'_{(x)} = 5\]
\[f'_{(x)} = u'(v_{(x)})\cdot v'_{(x)}\]
\[f'_{(x)}= e^{5x+1} \cdot 5 = 15 e^{5x+1}\]

\begin{definition}
Sei $f_{(x)} = e^{v_{(x)}} \Rightarrow f'_{(x)} = e^{v_{(x)} }\cdot v'(x)$

Die weiteren Ableitungsregeln gelten ebenso!
\end{definition}

\subsection{Eigenschaften von $f(x) = e^x $ und des Graphen von $f(x) = e^x$}
\begin {itemize}
\item Der Graph $G_t$ geht durch P(0|) ; f(0) = 1
\item Der Graph liegt immer über die x-Achse
\item Die x-Achse ist waagrechte Asymptote $f(x) \rightarrow 0$ für $x \rightarrow -\infty$
\item $f(x) \rightarrow + \infty \; ; \; x \rightarrow + \infty$
\item f ist auf $\mathbb{R}$ streng monoton wachsend
\item $G_f$ hat keine Extrem und Wendepunkte 
\[ \Rightarrow f'(x) \not = 0 \; ; \; f'' (x) \not = 0\]
\item $G_f$ ist Linksgekrümmt
\end{itemize}



\subsection{Darstellung der Euler Zahle}

Für $f(x) = e^x$ folgt aus : 
\[f'(0) = e^0 = 1\]
\[f'(x) = f(a) \cdot 1 \]
also:
\[f'(0) = \lim_{h\rightarrow0}\frac {e^h -1}h = 1\]

$\Rightarrow$ Wenn h sehr nahe bei 0 ist gilt:
\[\frac{e^h - 1}h \approx 1 \]
\[e^h-1 \approx h\]
\[ \Leftrightarrow e^h = h+1 \quad | \sqrt[h]{}\]
\[ e = (h+1)^{\frac1h}\]

Sei $n = \frac 1 h $ oder $h = \frac 1 n$
\[ \Rightarrow e  (1+ \frac 1 n )^n\]
für h $\rightarrow 0 $ geht $ n \rightarrow \infty4$
\[ \Rightarrow lim_{h \rightarrow \infty}(1+\frac 1 n )^n) = e\]

\[n = 10: (1+ \frac 1 {10} )^{10} \approx 2,59374246\]
\[n = 100: (1+ \frac 1 {100} )^{100} \approx 2,704813829\]
\[n = 1000: (1+ \frac 1 {1000} )^{1000} \approx 2,716923\]
\[n = 10000: (1+ \frac 1 {10000} )^{10000} \approx 2,718145927\]
\[n = 100000: (1+ \frac 1 {100000} )^{100000} \approx 2,718268237\]

\begin{definition}
\begin{itemize}
\item $f'(x) = f(x) $ oder $ \frac {f'(x) }{f(x)} = 1$
\item $f'(0) = 1$
\item $ \lim_{h\rightarrow0}\frac{e^h - 1}h = 1$
\item$\lim_{n\rightarrow \infty }(1+ \frac1 n )^n = e$
\end{itemize}
\end{definition}


\subsection{Lösen von natürlichen Exponentialgleichungen}
\begin{enumerate}
\item {Einfache Exponentialgleichungen: 
\[e^x = 5\]
\[ \Rightarrow ln(e^x) = ln(5)\]
\[ \Rightarrow x = ln(5) \approx 1,609\]
}

\item {Exponentialgleichungen, bei denen man ausklammern und den Satz von Nullpunkt anwenden kann:
\[x \cdot e^x + 3e^x = 0 \Rightarrow e^x \cdot(x+3) = 0\]
\begin{itemize}
\item Fall: $e^x = 0$: es existiert kein x
\item Fall: $ x + 3 = 0 \cdot x = -3$
\end{itemize}
Die Gleichung hat die Lösungmenge $\mathbb{L} = \{-3\}$
}

\item{Exponentialgleichungen, die sich durch Substitution lösen lassen:
\[e^{2x} - 5e^x+6 = 0\]
Ersetzen Sie $e^x$ durch u (Substitution) und lösen Sie die neu entstandene Gleichung:
\[e^x = u \Rightarrow u^2 = - 5u + 6 = 0 \Rightarrow u_1 = 3 \text{ oder } u_2 = 2\]
Ersetzen Sie u wieder durch $e^x$ (Rücksubstitution) und lösen Sie die neu entstanden Gleichung:
\[u = e^x \Rightarrow \]
\[ 3 = u_1 = e^{x_1} \Rightarrow x_1 = ln(3)\]
\[2 = u_2 = e^{x_2} \Rightarrow x_2 = ln(2)\]
Lösungsmenge: $\mathbb{L} = \{ln(2), ln(3)\}$
}
\end{enumerate}

Beispiel:

\[ \ln(5x^2) = 4 \ln(\sqrt x)+3 \quad \vert e^{()}\]
\[e^{ \ln(5x^2) } = e^ {4 \ln(\sqrt x)+3 }\]
\[5x^2 = e^3\cdot e^{4 \ln(\sqrt x)}\]
\[5x^2 = e^3\cdot e^{4 \ln(\sqrt x)}\]
\[5x^2 = e^3 \cdot e^{\ln(x^{\frac 42})}\]
\[5x^2 = x^2 \cdot e^3\]
\[x^2(5-e^3) = 0\]
\[\mathbb{L} = \{\} \text{ weil }x > 0 \text{ (Logarithmus is nur in } \mathbb{R}^+ \text{definiert})\]



\section{Exponentialfunktionen belibiger Basis}

\subsection{Anwendungen Vereinfachen von Termen}

\begin {enumerate}
\item $ e^{3ln(2)} = e ^{ln(2)^3} = 2^3 = 8$
\item $ln(\sqrt e ^3) = 3 ln(\sqrt e) = 3 \cdot 0,5 \cdot ln(e) = 1,5$ 
\item $ln(\frac 1 3 \sqrt e) = ln(\frac 1 3 ) + ln(\sqrt e) = ln(\frac 1 3 ) + 0,5 \cdot ln(e) = ln(1) - ln(3) + 0,5 $

\end{enumerate}




\subsection{Differenzierbarkeit von $f(x) = b^x$ auf der Stelle a}

Differenzquotient:
\begin{equation*}
\begin{aligned}
m(h) &= \frac {f(a+h) - f(h)}{h}
\\
&= \frac {b^{a+h}-b^a}{h} = \frac {b^a b^h - b^a}h
\\
&= \frac{b^a(b^h - 1)}h = b^a \cdot \frac {b^h - 1}h
\end{aligned}
\end{equation*}

\[m(h) = f(a) \cdot m_0(h)\]
\[m_0 = \text{Differenzquotient an der Stelle } 0\]


$\Rightarrow$ existierter Grenzwert $\lim_{h\rightarrow 0}\frac{h^h-1}h$, dann ist 
\[\lim_{h \rightarrow 0}m(h) = f(a) \cdot f'(0), \, f'(0) = \lim_{h \rightarrow0} \frac {b^h - 1}h\]

\begin{definition}
Für $f(x) = b^x \quad (b > 0 ; b \not = 1)$ gilt:
\begin {itemize}
\item f ist unendlich oft differenzierbar
\item $f'(a) =f(a) \cdot f'(0) $ oder $\frac {f'(a)}{f(a)} = f'(0) = h \quad h \in \mathbb{R}$
\end{itemize}

Für $f'(x)$ mit $f(x) = b^x$ gibt es keinen festen Term!
\end{definition}


\subsection{Umschreiben von Potenzen}

\[ a^b = e^{ln(a)^b} = e^{b \cdot ln(a)} = (e^{ln(a)})^b\]
$\Rightarrow$ Jede Potenz zu einer beliebigen Basis $a > 0$, laßt sich als Potenz zur Basis e schreiben!
\[ \mathcolorbox{yellow}{f(x) = 2^x = e^{ln(2^x)} = e^{x\cdot ln(2)} \quad f'(x) = e^{x \cdot ln(2)} \cdot ln(2)}\]

\subsection{Lösen von Exponentialgelichungen beleibiger Basis}
\begin{enumerate}
\item 
\[ 2^{\sqrt{x-1}}  = 3 \quad \vert \ln \text{ (Potenz schon isoliert)}\]
\[ln( 2^{\sqrt{x-1}}) = ln(3)\]
\[\sqrt{x-1}ln(2) = ln(3)\]
\[\sqrt{x-1} = \frac{ln(3)}{ln(2)}\]
\[x = (\frac{ln(3)}{ln(2)})^2 + 1\]

\item
\[4 \cdot  7^{2x-5} = 30 \quad \vert \text{Potenz isolieren}\]
\[7^{2x-5} = \frac {15}2 \quad \vert \ln \]
\[ln(7^{2x-5} ) = ln(\frac{15}2)\]
\[(2x-5)\cdot ln(7) = ln(7,5)\]
\[2x-5 = \frac{ln(7,5)}{ln(7)}\]
\[x = \frac{ln(7,5)}{ln(7)\cdot 2} + \frac 52\]

\item
\[2 ^{\sqrt{x-1}} = 3\]
\[2^{\sqrt{x-1}} = 2^{log_2(3)}\]
\[x-1 = log_2(3)^2\]
\[x = log_2(3)^2-1\]
\item
\[12+5^x = 25^x\]
\[12 = 25^x-5^x\]
\[12 = 5^{2^x} - 5^x\]
\[12 = 5^{x^2} - 5^x \quad \vert u = 5^x\]
\[u^2 - u - 12 = 0\]
\[(u-4)(u+3) = 0\]
\[5^x = 4 \rightarrow x = \log_5(4)\]
\[5^x = -3 \rightarrow \Lightning\]
\[ \mathbb{L} = \{ \frac{\ln(4)} {\ln(5)} \} \]

\end{enumerate}


\chapter{Verhalten für $x \rightarrow \pm \infty$}

\begin{definition}

Für die Funktionen der Form
\[f(x) = x^n\cdot e^{ax}\]
oder 
\[f(x) = x^n \cdot e^{-ax}\] mit $a>0$ gilt:

\[x^n \cdot e^{ax}\]
\begin{itemize}
\item 0 $\quad x\rightarrow \infty$ 
\item $+\infty \quad x \rightarrow + \infty$
\end{itemize}

\[ \frac{x^n}{e^ax} = x^n \cdot e^{-ax}\]
\begin{itemize}
\item $+\infty \quad x \rightarrow - \infty$ n gerade
\item $0 \quad x \rightarrow + \infty$

\end{itemize}

\[ \frac{x^n}{e^ax} = x^n \cdot e^{-ax}\]
\begin{itemize}
\item $-\infty \quad x \rightarrow - \infty$ n ungerade
\item $0 \quad x \rightarrow + \infty$
\end{itemize}

\[e^{ax} - x^n\]
\begin{itemize}
\item $-\infty \quad x \rightarrow + \infty$ n ungerade
\item $-\infty \quad x \rightarrow - \infty$ n gerade
\item $+ \infty \quad x \rightarrow + \infty$
\end{itemize}

Der Satz gilt auch wenn $x^n$ durch ein Polynom $p(x)$ von grad n ersetzt

\end{definition}

\begin{definition}
Eine Gerade mit der Gleichung $ y = c$ heißt
\underline{waagenrechte Asymptote des Schaubildes von $f$, wenn gilt:} 
\[ \lim_{x \to \infty} [ f(x) - d] = 0\]
bzw. $f(x) \rightarrow d$  für $x\rightarrow + \infty$
(analog für  $x \rightarrow - \infty$)
\end{definition}

\underline{Beispiel:}
\[f(x) = e^x + 2\]
\[\lim x {-\infty} [f(x) - 2] = \lim_{x \to - \infty} e^x = 0\]
oder $e^x + 2 \rightarrow 2 $ für $ x \rightarrow - \infty $ weil $e^x \rightarrow 0 $ für $ x \rightarrow - \infty$
$ \Rightarrow y = 2 $ ist Waagenrechte Asymptote

\begin{definition}
Gegeben ist eine Funktion f mit dem Definitionsbereich $D_f$. Wenn für alle $x \in D_f$ gilt, so ist der Graph von f
\[f(-x) = f(x) \; \Rightarrow \text{achsensymmetrisch zur y-Achse}\]

\[ f(-x) = -f(x) \; \Rightarrow \text{punktsymmetrisch zum Ursprung}\]

\end{definition}

\underline{Wiederholung: }
Symetrisch bei generationalen Funktionen
\[f(x) = a_n \cdot x^n + \dots +a_1 \cdot x + a_0\]

\underline{Achsensymmetrie zur y=Achse $\Leftrightarrow$ alle Hochzahlen sinf gerade}
z.B. \[f(x) = x^4 + 4x^2 + 7\]

\underline{Punktsymetrie zur Ursprung $ \Leftrightarrow $ alle Hochzahlen sind ungerade}
z.B. \[f(x) = -7x^5 - 3x^3 + x\]

\chapter{Überprüfen von Symetrie}

\begin{enumerate}
\item
\[f(x) = x^2\]
\[f(-x) = (-x)^2 = x^2 = f(x)\]
$\Rightarrow$ y-Achsensymetrisch

\item
\[ g(x) = x^3\]
\[g(-x) = (-x)^3 = -x^3 = -g(x)\]
$ \Rightarrow$ Punktsymetrisch zum Ursprung
\item
\[h(x) = \frac{x} {x^3 + x}\]
\[h(-x) = \frac{-x}{(-x)^3+(-x)} = \frac{-x}{-x^3-x} = \frac{-x}{-(x^3+x)} = \frac x {x^3+x} = h(x)\]
\[-h(x) = -\frac{x} {x^3 + x}\]
$\Rightarrow$ y-Achsesymmetrisch

\item
\[i(x) = x^3 \cdot e^{x^2}\]
\[i(-x) = (-x)^3 \cdot e^{(-x)^2} = -x^3 \cdot e^{x^2} = -i(x)\]
$\Rightarrow$ Punktsymetrisch zum Ursprung

\item
\[ j(x) = e^x + x^2\]
\[j(-x) = e^{-x} + (-x)^2 = e^{-x} + x^2\]
\[-j(x) = -( e^x + x^2) = -e^x - x^2\]
\[j(x) \not = j(-x) \land j(-x) \not = -j(x)\]
$\Rightarrow$ keine Symmetrie

\end{enumerate}

\chapter{Untersuchung auf Eigenschaften}
\begin{enumerate}
\item Nullstellen
\item Extremstellen / Monotonie
\item Wendestellen / Krümmung
\item Symmetrie
\item Asymptoten
\end{enumerate}

\underline{Beispiel:}
\[f(x) = x^4 \cdot e^x\]
Nullstellen:
\[x^4 = 0 \quad e^x = 0\]
\[ \mathbb{L} = \{0\}\]
Extremstellen:
\[f'(x) = 0 = 4x^3\cdot e^x + x^4 \cdot e^x = x^3\cdot e^x(4 + x)\]
\[f''(x) = 12x^2 \cdot e^x + 4x^3 \cdot e^x + 4x^3\cdot e^x + x^4 \cdot e^x\]
\[f''(0) = 12 \cdot 0^2 \cdot e^0 + 4\cdot 0^3 \cdot e^0 + 4\cdot 0^3\cdot e^0 + 0^4 \cdot e^0= 0\]
\[f''(0) \text{ hat Vorzeichenwechsel } - \to +\]
\[f''(-4) = 12\cdot (-4)^2 \cdot e^{-4} + 4\cdot(-4)^3 \cdot e^{-4}+ 4\cdot(-4)^3\cdot e^{-4} + (-4)^4 \cdot e^{-4}\not = 0\] 
\[ \mathbb{L} = \{ 0; -4\}\]
Monotonie:
\[I_1 = ]-\infty;-4[ \quad I_2 = ]-4;0[ \quad I_3 = ]0; \infty[\] 
\[f'(-6) = + \ldots \rightarrow I_1\text{ist streng monoton wachsend}\]
\[f'(-2) = - \ldots \rightarrow I_2\text{ist streng monoton fallend}\]
\[f'(1) = + \ldots \rightarrow I_3\text{ist streng monoton wachsend}\]
Wendestellen:
\[f''(x) = 0 = 12x^2 \cdot e^x + 4x^3 \cdot e^x + 4x^3\cdot e^x + x^4 \cdot e^x = x^4 \cdot e^x + 8x^3 \cdot e^x + 12x^2 \cdot e^x\]
\[f''(x)= e^x \cdot x^2 (x^2 + 8x +12) = e^x \cdot x^2 (x + 2)(x+6)\]
\[f'''(x) = [x^4 \cdot e^x + 8x^3 \cdot e^x + 12x^2 \cdot e^x]' = 4x^3 \cdot e^x + e^x \cdot x^4 + 8(3x^2 \cdot e^x + e^x \cdot x^3) + 12(2x\cdot e^x+e^x\cdot  x^2)\]
\[f'''(x) = e^x x^4 + 12e^xx^3 + 36 e^xx^2 + 24e^xx\]
\[f'''(0) = 0\]
\[f'''(-2) \not = 0\]
\[f'''(-6)\not = 0\]
\[\mathbb{L} = \{-6;-2;0\}\]
Krümmung:
\[f''(-10) = +\ldots \; \rightarrow \; ]\infty;-6[ \text{ ist linksgekrümmt}\]
\[f''(-5) = -\ldots \; \rightarrow \;]-6;-2[ \text{ ist rechtsgekrümmt}\]
\[f''(-1) = +\ldots \; \rightarrow \;]-2;\infty[ \text{ ist linksgekrümmt}\]
Symetrie:
\[f(x) = x^4 \cdot e^x\]
\[f(-x) =  (-x)^4 \cdot e^{-x} = x^4 e^{-x} \not = f(x)\]
\[-f(x) = - x^4 \cdot e^x \not = f(-x)\]
\[ \Rightarrow \text{keine Symetrie}\]

Asymptoten:
\[x \to +\infty \; \rightarrow f(x) \to \infty \] 
\[x \to -\infty \; \rightarrow f(x) \to 0\]
\[\Rightarrow \text{Asymptoten: } \{y = 0\} \]


\end{document}
