\documentclass{book}
\usepackage[shortlabels]{enumitem}
\usepackage{amsmath}
\usepackage{amssymb}
\usepackage{booktabs}
\usepackage[most]{tcolorbox}

\tcbset{colback=yellow!10!white, colframe=red!50!black, 
        highlight math style= {enhanced, %<-- needed for the ’remember’ options
            colframe=red,colback=red!10!white,boxsep=0pt}
        }
\usepackage{marvosym}
\let\marvosymLightning\Lightning

\usepackage{pgfplots}
\pgfplotsset{compat=newest}
\usepgfplotslibrary{fillbetween}

\newcommand{\mathcolorbox}[2]{\colorbox{#1}{$\displaystyle #2$}}

\newcommand*\circled[1]{\tikz[baseline=(char.base)]{
\node[shape=circle, draw, inner sep=2pt] (char) {#1};}}

\newtheorem{definition}{Definition}

\begin{document}

\begin{definition}[Zusammenfassung von Sinus und Cosinus Funktion]
    
    \begin{displaymath}
        G_{sin} \to G_{cos}: \text{Verschiebung um } -\frac {\pi}{2}LE 
    \end{displaymath}
    
\end{definition}

\section{Funktionsanpassung mit trigonometrischen Funktionen}
    anpassung 
    \section{Trigonometrische Gleichungen}
    Es gilt:
    \begin{enumerate}[a)]
        \item Werte den Sinusfunktion und Kosinusfunktion
            \begin{table}[h!]
                \begin{tabular}{c||*{5}{c|}}
                    x & 0 & $\frac {\pi}6$ &$\frac {\pi}4$ &$\frac {\pi}3$ &$\frac {\pi}2$ \\
                    \midrule
                    $\sin x$ & 0 & $\frac 12$ & $\frac{\sqrt{2}}2$ & $\frac{\sqrt{3}}2$ & 1 \\
                    $\cos x$ & 1 & $\frac{\sqrt{3}}2$ & $\frac{\sqrt{2}}2$ & $\frac 12$ & 1 \\

                \end{tabular}
            \end{table}
        \item Alle Lösungen mit $x \in [0;2\pi]$
            \[\sin x = \frac 12 \sqrt 2 \quad \Rightarrow x_1=\frac \pi 4 \: \text{ oder} \; x_2 = \pi - \frac \pi 4 = \frac 34 \pi\]
            \[\cos x = -\frac {\sqrt{2}} 2 \quad \Rightarrow x_1 = \pi - \frac \pi 4 = \frac 34 \pi \; \text{oder}\; x_2 = \pi+\frac \pi 4 = \frac 54 \pi\]
            
        \item Wichtige Hilfsmittel für weitere Gleichungen
            \begin{enumerate}[i)]
                \item Substitution des Arguments
                    \[\sin (3x) = \frac 12 \quad \]
                    Substitution (z = 3x):
                    \[\sin(z) = \frac 12 \quad \Rightarrow z_1 = \frac \pi 6 \; \text{oder} \; z_2 = \pi - \frac \pi 6 = \frac 56 \pi\]
                    Rücksubstitution:
                    \[x_1 = \frac {z_1}3 = \frac \pi {18}\]
                    \[x_2 = \frac {z_2}3 = \frac 5{18}\pi\]
                \item trigonometrische Hilfsmittel \\
                    \tcboxmath{
                        \tan x = \frac {\sin x} {\cos x} 
                    }
                    \tcboxmath{
                        \sin^2 x +\cos ^2 x = 1 
                    }

                Tangensfunktion Definitionsgleichung:
                \begin{align*}
                    2\sin x = \tan x &\Rightarrow 2 \sin x  = \frac {\sin x }{\cos x} \\
                                     &\Leftrightarrow 2\sin x - \frac {\sin x }{\cos x} = 0 \\
                                     &\sin x (2 - \frac 1{\cos x} = 0 \\
                                     & \Rightarrow \sin x = 0 \; \text{oder} \; \cos x = \frac 12 \\
                                     &&& x_1 = 0 \\
                                     &&& x_2 = \pi \\
                                     &&& x_3 = \frac \pi 3 \\
                                     &&& x_4 = \frac 35 \pi
                \end{align*}

                \item Substitution der Winkelfunktion
                    \[\sin^2 x + \frac 32 \sin x - 1 = 0 \]
                    Substitution z = $\sin x$: 
                    \[z^2 \frac 32 z -1 = 0\]
                    \[(z-\frac 12)(z+2)=0\]
                    \[z_{1,2} = \frac 12 \; ; \; -2\]
                    Rücksubstitution:
                    \[\sin x = -2 \]
                    \centering \Lightning
                    \[\sin x = \frac 12 \]
                    \[x_1 = \frac \pi 6 \; \text{oder} \; x_2 = \pi - \frac \pi 6 = \frac 56 \pi\]


            \end{enumerate}
        
\end{enumerate}
    



\end{document}
