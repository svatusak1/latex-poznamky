\documentclass{book}
\usepackage{multicol}
\usepackage[shortlabels]{enumitem}
\usepackage{amsmath}
\usepackage{amssymb}
\usepackage{mathtools}
\usepackage{marvosym}
\let\marvosymLightning\Lightning

\usepackage{pgfplots}
\pgfplotsset{compat=newest}
\usepgfplotslibrary{fillbetween}

\newcommand{\mathcolorbox}[2]{\colorbox{#1}{$\displaystyle #2$}}

\newcommand*\circled[1]{\tikz[baseline=(char.base)]{
\node[shape=circle, draw, inner sep=2pt] (char) {#1};}}

\newtheorem{definition}{Definition}

\begin{document}

\begin{definition}
    $G_k$ ist punktsymetrisch g.d.w. alle Exponenten ungerade sind
\end{definition}

\begin{definition}
    Seien f eine ganzrationale Funktion mit $f(x) = (x-a)^k \cdot g(x)$ und g eine ganzrationale Funktion mit $g(a) \not = 0$ 
    und $k \in \mathbb{N} \ {0}$, dann gilt:
    \begin{itemize}
        \item \underline{k = 1}: $G_f$ schneidet die x-Achse an x=a
        \item \underline{k gerade}: $G_f$ hat für $x=a$ eine Exremstelle auf der x-Achse
        \item \underline{k underade} ($k \not = 1$): $G_f $ hat für x = a eine Sattelstelle auf der x-Achse
    \end{itemize}
\end{definition}

\chapter {Lösen von Gleichungen}

\section{Strategien}

\begin{enumerate}[a)]
    \item Quadratische Gleichunge $ax^2+bx+c=0$ 
        Lösungsformel: 
        \begin{itemize}
            \item \[x_{1, 2} = \frac{-b\pm\sqrt{b^2-4ac}}{2a}\]
            \item \[x_{1, 2} = -\frac p 2 \pm \sqrt{(\frac p 2 )^2-q}\]
        \end{itemize}
    \item Produkt $ p(x) \cdot q(x) = 0$ 
        Satz von Nullprodukt: $ p(x) = 0$ oder $q(x) = 0$
    \item Jeder Summant enthält x $ ax^3 + bx^2+cx = 0$
        Ausklammern: $ x\cdot (ax^2+bx+c)=0$
    \item Gleichungen, die durch Substitution gelöst werden können 
        \begin{multicols}{2}
            \[ a\cdot e^{2x} +b \cdot e^x +c = 0\]
            \[ax^4 + bx^2 +c = 0\]
            \[a(sin(x))^2 + b\cdot sin(x) + c = 0\]

            \columnbreak
            \begin{equation*}
                \begin{rcases}
                    \begin{array}{c}
                        z=e^x \\
                        z = x^2 \\
                        z = sin(x)
                    \end{array} 
                \end{rcases}
                \text{Entries}
            \end{equation*}
        \end{multicols}

\end{enumerate}

\chapter {Die Sinus und Kosinus Funktion}

\begin{definition}
    Die eindeutige Zuordnung, die jeder reelen Zahl $x\in\mathbb{R}$ (im Bogenmaß) ($D_f = D_g = \mathbb{R}$) den Wert
    \begin{itemize}
        \item sin x zuordnet heißt Sinusfunktion, $f(x)=sin x$
        \item cos x zuordnet heißt Kosinusfunktion $g(x) = cos x$
    \end{itemize}
    f und g gehören ze den trigonometrischen Funktionen
\end{definition}

\section{Amplitude und Periode der Sinusfunktion}
\begin{enumerate}
    \item $f(x) = sin (x)$ hat die Periode $p = 2 \pi$
    \item $g(x) = cos (x)$ hat die Periode $p = \pi$
    \item $h(x) = sin (bx)$ hat die Periode $p = \frac {2\pi}b$
\end{enumerate}

\end{document}
