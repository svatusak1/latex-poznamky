\documentclass{book}
\usepackage[shortlabels]{enumitem}
\usepackage{amsmath}
\usepackage{amssymb}
\usepackage{marvosym}
\let\marvosymLightning\Lightning

\usepackage{pgfplots}
\pgfplotsset{compat=newest}
\usepgfplotslibrary{fillbetween}

\newcommand{\mathcolorbox}[2]{\colorbox{#1}{$\displaystyle #2$}}

\newcommand*\circled[1]{\tikz[baseline=(char.base)]{
\node[shape=circle, draw, inner sep=2pt] (char) {#1};}}

\newtheorem{definition}{Definition}

\begin{document}

\begin{definition}[Stecken und Verschieben von Graphen]

    Gegeben ist der Graph $G_f$ der Funktion g mit 
    \[
        g(x) = a \cdot f(c\cdot (x-c)) +d
    \]
    \[ a, b > 0 c, d, \in \mathbb{R}\]
    entsteht aus dem Graphen $G_f$ von f durch:
    \begin{enumerate}[i)]
        \item Strekung in y-Richtung mit dem Faktor a
        \item Streckung in x-Richtung mit dem Fakto $\frac 1b$
        \item Verschiebung in pos. x-Richtung um c LE 
        \item Verschiebung in pos. y-Richtung um d LE 
    \end{enumerate}

    für $ a<0$ findet zusätzlich zur Strekung in y-Richtung eine Spiegelung an der x-Achse statt.
    für $b<0$ findet zusätzlich zur Strekung in x-Richtung eine Spiegelung an der y-Achse.
    
\end{definition}

\begin{definition}[llgemeine Sinusfunktion]
    
\end{definition}


\end{document}
