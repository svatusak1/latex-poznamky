\documentclass{book}

\usepackage{amsfonts}
\usepackage{amsmath}
\usepackage{mathtools}
\newtheorem{definition}{Definition}

\begin{document}

\section{Graphen von Exponentialfunktionen mit Parameter}
\[e^x - a (a\in \mathbb{R})\]
Eine Erhöhung von a bewirkt eine Verschiebung des Graphen $G_f$ im Richtung die negative y-Achse.

\section{Umkehrbarkeit und Umkehrfunktion}

\begin{definition}
Eine Funktion $f$ mit der Definitionsmenge $D_f$ und der Wertemenge $W_f$ ist heißst \textbf{umkehrbar}, wenn es zu jedem $ y \in W_f$ genau ein $x \in D_f$ mit $ f(x) = y$ gibt.
\end{definition}

\begin{definition}
Jede streng monotone Funktion $f$ ist umkehbar.
Insobesondere ist jede in einem Intervall $I$ differenzierbare Funktion $f$ mit \[f'(x) > 0\text{(bzw. $ f'(x) < 0$)}\] für alle $x \in I $ umkehrbar.
\end{definition}

\end{document}