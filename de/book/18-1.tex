\documentclass{article}
\usepackage[shortlabels]{enumitem}
\usepackage{amsmath}
\usepackage{amssymb}
\usepackage{stmaryrd}
\usepackage{siunitx}
\usepackage{pgfplots}
\pgfplotsset{compat=1.18}


\newtheorem{definition}{Definition}

\begin{document}
\chapter{Integralen}
\section{Das Integral als orientierte Flacheinhalt}

\subsection{Beispiel für naherungsweise Berechnung von Flacheinhalt}
    
    
\begin{tikzpicture}
\begin{axis}[
	axis lines = left,
	xlabel = \(x\),
	ylabel = \(y\),
	xtick distance = 0.5,
	ytick distance = 1,
	title = {Untersumme},
]
\addplot[
	color=black,
	domain = 0:2,
	]{x^2};
	
\draw (axis cs: 0,0) -- (axis cs: 0.5,0)[color = red];
\draw (axis cs: 0.5,0) rectangle (axis cs: 1,0.25)[color = red];
\draw (axis cs: 1,0) rectangle (axis cs: 1.5,1)[color = red];
\draw (axis cs: 1.5,0) rectangle (axis cs: 2,2.25)[color = red];
\end{axis}
\end{tikzpicture}

\begin{definition}
sei $[a;b] \leq D_f$ mit $f(x) \geq 0$ (Untersumme) f.a. $x \in [a;b]$
Man erhält den \underline{Flächeinfalt zwichen $G_f$ und der x-Achse über [a;b]} näherungsweise mit \underline{Stützstelle} und \underline{Intervalbreite}
\[A \approx S_n = \overbrace{\Delta x}^{\text{Intervalbreite}} \overbrace{(f(\Delta x \cdot 0)+\dotsc + f(\Delta x \cdot (n-1)))}^{\text{Stützstelle} }\text{ mit } \Delta x =\frac {b-a}n\]
\end{definition}

\begin{tikzpicture}
\begin{axis}[
	axis lines = left,
	xlabel = \(x\),
	ylabel = \(y\),
	xtick distance = 0.5,
	ytick distance = 1,
	title = {Obersumme},
]
\addplot[
	color=black,
	domain = 0:2,
	]{x^2};
	
\draw (axis cs: 0,0) rectangle (axis cs: 0.5,0.25)[color = green];
\draw (axis cs: 0.5,0) rectangle (axis cs: 1,1)[color = green];
\draw (axis cs: 1,0) rectangle (axis cs: 1.5,2.25)[color = green];
\draw (axis cs: 1.5,0) rectangle (axis cs: 2,4)[color = green];
\end{axis}
\end{tikzpicture}

\[A \approx S_n = \Delta x (f(\Delta x \cdot 1)+\dotsc + f(\Delta x \cdot n)) \text{ mit } \Delta x = \frac {b-a}n\]

\begin{tikzpicture}
\begin{axis}[
	axis lines = left,
	xlabel = \(x\),
	ylabel = \(y\),
	xtick distance = 0.5,
	ytick distance = 1,
	title = {Rechtecksumme},
]
\addplot[
	color=black,
	domain = 0:2,
	]{x^2};
	
\draw (axis cs: 0,0) rectangle (axis cs: 0.5,0.0625)[color = blue];
\draw (axis cs: 0.5,0) rectangle (axis cs: 1,0.5625)[color = blue];
\draw (axis cs: 1,0) rectangle (axis cs: 1.5,1.5625)[color = blue];
\draw (axis cs: 1.5,0) rectangle (axis cs: 2,3.0625)[color = blue];
\end{axis}
\end{tikzpicture}

\[A \approx S_n = \Delta x (f(\Delta x \cdot 0+ \Delta x \frac 12)+\dotsc + f(\Delta x \cdot (n-1)+\Delta x \frac 12)) \text{ mit } \Delta x = \frac {b-a}n\]


\end{document}