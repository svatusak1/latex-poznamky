\documentclass{book}
\usepackage[shortlabels]{enumitem}
\usepackage{amsmath}
\usepackage{amssymb}
\usepackage{marvosym}
\let\marvosymLightning\Lightning

\usepackage{pgfplots}
\pgfplotsset{compat=newest}
\usepgfplotslibrary{fillbetween}

\newcommand{\mathcolorbox}[2]{\colorbox{#1}{$\displaystyle #2$}}

\newcommand*\circled[1]{\tikz[baseline=(char.base)]{
\node[shape=circle, draw, inner sep=2pt] (char) {#1};}}

\newtheorem{definition}{Definition}

\begin{document}

\begin{definition}
    Eine Funktion $f(x) = a_n x^4 + a_{n-1} x^{n-1} + \dots + a_1 x+a_0 \; (a_n \not = 0 , n \in \mathbb{N}_0)$
    deren Funktionsterm ein Polynom ist oder auf diese Form gebracht werden kann, heißt \underline{ganzrationale Funktion
    von Grad n}. Das Schaubild heißt für $n \geq 2 $ \underline{Parabel n-ter Ordnung}
\end{definition}

\[f(x) = 2 (x+1)(x-2)(3-x) = -2(x^2-x-2)(x-3) = -2 (x^3 - 3 x^2 -x^2 +3x - 2x +6) = -2x^3 + 8x^2 -2x -12\]
$\Rightarrow$ es ist eine ganzrationale Funktion

\begin{definition}
    Eine Zahl $ x\in D_f$ heißt \underline{Nullstelle der Funktion f}, wenn gilt \underline{f(x) = 0}.
    Das Schaubild $G_f$ schneidet die x-Achse im $N(x|0)$
\end{definition}


\begin{definition}
Sei f eine ganzrationale Funktion vom Grad n und $x_1$ eine Nullstelle von $f$, dann gibt es eine ganzrationale Funktion g von Grad n-1 mit $f(x) = (x-x_1) \cdot g(x)$
\end{definition}
\begin{definition}
Eine ganzrationale Funktion f von Grade n hat höhstens n Nullstellen
\end{definition}

\begin{definition}
    Eine ganzrationale Funktion von Grad n hat höhstens n-1 Extremstellen.
    Weil Ableitung eine ganzrationale Funktion den Grad um 1 vermindet.
\end{definition}
\begin{definition}
    Eine ganzrationale Funktion von Grad n hat höhstens n-2 Wendestelln
    Weil zweite Ableitung eine ganzrationale Funktion den Grad um 2 vermindet.
\end{definition}

\end{document}
