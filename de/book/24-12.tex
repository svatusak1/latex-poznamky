\documentclass{article}
\usepackage[shortlabels]{enumitem}
\usepackage{amsmath}
\usepackage{amssymb}
\usepackage{marvosym}
\let\marvosymLightning\Lightning

\begin{document}

\underline{\textbf{Seite 65:}}

\underline{\textit{\textbf{9a)}}}
bestimmen Sie einen Funktionsterm der Umkehrfunktion von f
\[f(x) = x^2; \; D_f ] -\infty ; 0 ]\]

\[y = x^2\]
\[x = \pm \sqrt y\]
\[\bar f = \pm \sqrt y\]
\[D_f \textrm{ von } \bar f = H_f \textrm{ von } f \Rightarrow \bar f = -\sqrt y\]

\underline{\textit{\textbf{10c)}}}
Zeigen Sie mithilfe der Ableitung, dass die Funktion f umkehrbar ist
\[f(x) = (x^2 + \frac 3 {16}) \cdot e^{-2x} \; ; \; D_f = [0,25 ; 0,75]\]

\[f'(x) = 2x \cdot e^{-2x} + (x^2 + \frac 3 {16}) \cdot (-2)e^{-2x} = 2e^{-2x} (x - (x^2 + \frac 3 {16}) )\]
\[f'(x) = 0 \]
\[(x - (x^2 + \frac 3 {16}) ) = -x^2+x - \frac 3 {16}= 0\]
\[x_{1, 2} = \frac{-1 \pm \sqrt{1-4 \cdot (-1)\cdot(- \frac 3 {16})}}{-2}\]
\[x_{1, 2} = \frac 1 4 \; ; \; \frac 34\]
Funktion f ist in $ D_f$ entweder streng monoton wachsend oder s.m.f. $\Rightarrow$ ist in $D_f$ umkehrbar

\underline{\textit{\textbf{14}}}

Die Funktion f mit $f(t) = 5 \cdot \sqrt{t+1}$ beschreibt für $t \ge 0$ die jährlichen Verkaufszahlen eines Elektro-Modells (in Tausend Stück, t in Jahren nach Markteinführung)
\begin{enumerate}[a)]

\item Nach wie vielen Jahen werden jährlich 11 000 Stück dieses E-Autos verkauft?

\[f(t) = 11 000 = 5 \cdot \sqrt{t+1}\]
\[2200 =\sqrt{t+1}\]
\[4.840.000 = t + 1\]
\[4.839.999 = t\]

\item Beschreiben Sie in Worten, was man in diesem Sachzusamenhang mithilfe der Umkehrfunktion $\bar f$ von f berechnen kann, und bestimmen Sie einen Funktionsterm von $\bar f$
\\
mit die Umkehrfunktion kann man berechnen im welcher Jahr die Menge x von E-Autos verkaufen war.

\[y = 5\sqrt{t+1}\]
\[t = \frac{y^2}{25}-1\]
\[\bar f = \frac{x^2}{25}-1\]

\item Formulieren sie im Sachzusmenhang eine Frage zur Gleichung $\bar f (2y) - \bar f (y) = 6$. Drücken Sie diesen Sachverhalt durch Gleichung mit f statt $\bar f $ aus.
\\
Mit welcher Menge von Autos dauert es noch 6 Jahren um die Menge von Autoverkauf zu verdoppeln.
In welcher Jahr wird nach genauso Jahren  sechs mehr autos gemacht.

\end{enumerate}

\underline{\textbf{Seite 69:}}

\underline{\textit{\textbf{7}}}

Gegeben ist die Funktion f mit $ f(x) = 3 \ln(4x+6)$
\begin{enumerate}[a)]
\item Geben Sie dei maximale Definitionsmenge und die Wertemenge von f an

\[D_f = ]-\frac 32;\infty[\]
\[W_f = \mathbb{R}\]

\item Zeigen Sie, dass f umkehrbar ist, und bestimmen Sie einen Term der Umkerhfunktion $\bar f$

\[f'(x) = 3 \frac 1 {4x+6} \cdot 4 = \frac 6 {2x+3}\]
\[f'(x) = 0 \; \Lightning\]
\[ f'(x) > 0 \text{ für } x > -\frac 32\]
\[f'(x) < 0 \text{ für } x < -\frac 32\]
\[D_f = ]-\frac 32;\infty[ \Rightarrow \text{ f ist umkehrbar}\]

\[y = 3 \ln(4x+6)\]
\[\frac y3 = \ln(4x+6)\]
\[e^{\frac y3} = 4x+6\]
\[x = \frac {e^{\frac y3}-6}4\]
\[\bar f = \frac {e^{\frac x3}-6}4\]

\item Geben Sie die Definitionsmenge und Wertemenge von $\bar f$ an

\[D_f = \mathbb{R}\]
\[W_f = ]-\frac 32;\infty[\]

\end{enumerate}



\underline{\textit{\textbf{11}}}

Die funktion h mit $h(t) = 4\frac 1{(t+1)^2}\cdot \ln(t+1)$ beschreibt die Höhe der Hochwassers in einem Ort (t in Tagen nach Beginn des Regens, h(t) in m über Normal)

\begin{enumerate}[a)]

\item Bestimmen Sie den höhsten Wassestand über Normal

\[f'(x) = 4 \cdot ((t+1)^{-2}\cdot \ln(t+1))'\]
\[f' = 4 (-2(t+1)^{-3}\ln(t+1) + \frac 1{t+1} (t+1)^{-2}) = 4(t+1)^{-3}(1-2\ln(t+1))\]
\[f' = 0 \rightarrow 1-2\ln(t+1) = 0\]
\[\frac 12=\ln(t+1)\]
\[e^{\frac12} = t+1\]
\[t = e^{\frac12}-1 = \sqrt e -1\]
\[f'(0) = 4 \Rightarrow \text{ bei } \sqrt e -1 \text{ gibt es Zeichnungswechsel } + \to - \rightarrow \text{ es ist das Maximum}\]

\item Berechnen Sie, zu welchen Zeitpunkt das Hochwasser am stärksten abnimmt

\[f''(x) = [4(t+1)^{-3}(1-2\ln(t+1))]' \]
\[f''(x) = 4(-3(t+1)^{-4}(1-2\ln(t+1)) + (t+1)^{-3}(-2)\frac1{t+1}) \]
\[f''(x) = 4\frac{6\ln(t+1)-5}{(t+1)^4}\]
\[f''(x) = 0 \rightarrow 6\ln(t+1)-5=0\]
\[6\ln(t+1)=5\]
\[t = e^{\frac 56}-1\]

\item Ermitteln Sie, mit welcher Hochwasserhöhe man langfristig rechnen kann

\[ f \to \infty\]
\[x \to 0\]
weil die hoch -2 übernimmt der Logarithmus

\item Interpretieren Sie die Gleichung $ h(t+2) - h(t) = -0,2$, im Sachzusammenhang
\\
in welchen zwei Tagen nimmt das Wasser um 20 Zentimetern ab

\item Nach vier Tagen geht die Abnahme des Hochwassers ohne Knick in eine lineare Abnahme über. Berechnen Sie, wann weider die normale Wasserhöhe erwartet wird.

\[f'(4) = 4(4+1)^{-3}(1-2\ln(4+1)) = -0,0710040263958\]
\[y = -0.0710040263958(x-4)+f(4) = -0.0710040263958(x-4)+0.2575100659895\]
\[-0.0710040263958(x-4)+0.2575100659895 = 0\]
\[x = 7,62669\]

\end{enumerate}

\underline{\textit{\textbf{1c}}}

Bestimmen Sie die Ableitung von f

\[f(x) = 3\ln(x+1)\]
\[f'(x) = 3 \frac 1 {x+1}\]

\underline{\textit{\textbf{3}}}
Geben Sie die Nullstellen der Funktion f an

\begin{enumerate}[a)]
\item[c]
\[f(x) = \ln  x +3\]
\[f'(x) = \frac 1 x +3\]
\[f'(x) = 0 \Lightning \]

\item[e]
\[f(x) = \ln(x^2 - 3) \; D_f = \mathbb{R} \\ <-\sqrt3; \sqrt3>\]
\[f'(x) = \frac {2x}{x^2 - 3}\] 
\[f'(x) = 0\]
\[x = 0 \; ; \; x \not \in D_f \Rightarrow \text{Nullstellen = }\emptyset\]
\end{enumerate}

\underline{\textit{\textbf{5}}}
Bestimmen sie die Ableitung und vereinfachen Sie, wenn möglich
\begin{enumerate}[a)]
\item[c]

\[f(x) = \ln(4(x-2)^3)\]

\[f'(x) = \frac {12(x-2)^2}{4(x-2)^3} = \frac 3 {x-2}\]

\item[g]

\[f(x) = x^2\cdot (\ln(x) - \frac 12)\]

\[f'(x) = 2x (\ln(x) - \frac 12) + \frac {x^2} x = 2x\ln(x) - x + x = 2x \ln (x)\]
\end{enumerate}

\underline{\textit{\textbf{8: 3}}}

Gegeben ist die Funktion mit:
\[f(x) = 4e^{3x-2}+1\]
\begin{enumerate}[a)]

\item Geben Sie jeweils die maximale Definitionsmenge und die Wertemenge der Funktion f an

\[D_f = \mathbb{R}\]
\[W_f = (1;\infty)\]

\item Zeigen Sie jeweils, dass f umkehrbar ist, und bestimmen Sie einen Term der Umkehrfunktion $\bar f$ an

\[f'(x) = 12 e^{3x-2} > 0 \Rightarrow \text{ f ist s.m.w. } \Rightarrow \text{ umkehrbar}\]
\[y = 4e^{3x-2}+1\]
\[\frac {y-1}4 = e^{3x-2}\]
\[\ln (\frac {y-1}4) = 3x-2\]
\[x = \frac {\ln (\frac {y-1}4) +2}3\]
\[\bar f = \frac {\ln (\frac {x-1}4) +2}3 = \frac {\ln (x-1)-\ln(4) +2}3\]

\end{enumerate}

\underline{\textit{\textbf{9}}}
Geben Sie den Grenzwert an

\begin{enumerate}[a)]
\item[b]

\[ \lim_{x \to 0} ((x^2+2x)\ln(x)) = 0\]

\item[c]

\[ \lim_{x \to \infty} \frac {\ln(x)}{3x^2-2x-4} = 0\]
\end{enumerate}

\underline{\textit{\textbf{10c}}}

Bestimmen Sie die Koordinaten der Extrempunkte des Graphen von f

\[f(x) = x^2\ln(x)\]

\[f'(x) = 2x\ln x + x^2\frac1x = x (2\ln x + 1) \]
\[f'(x) = 0 \; x = 0 \; ; \; 2 \ln x +1 = 0\]
\[2 \ln x +1 = 0\]
\[x = e^{-\frac12} = \frac 1 {\sqrt e}\]

\[f(0) \Lightning\]
\[f(\frac 1 {\sqrt e}) = (\frac 1 {\sqrt e})^2\ln(\frac 1 {\sqrt e}) = \frac 1 e \cdot ( 0 - \ln{\sqrt e}) = -\frac {\ln \sqrt e}{e}\]

\[\mathbb{L} = \{ [\frac 1 {\sqrt e };-\frac {\ln \sqrt e}{e}]\}\]






















\end{document}