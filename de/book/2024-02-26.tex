\documentclass{article}
\usepackage[shortlabels]{enumitem}
\usepackage{amsmath}
\usepackage{amssymb}
\usepackage{marvosym}
\let\marvosymLightning\Lightning

\usepackage{pgfplots}

\newcommand{\mathcolorbox}[2]{\colorbox{#1}{$\displaystyle #2$}}

\newtheorem{definition}{Definition}

\begin{document}

\begin{definition}

Gegeben ist eine über dem Intervall [a;b] integrierbare Funktion f. Rotiert die Fläche zwischen dem Graphen von f und der x-Achse über dem Intervall [a;b] um die x-Achse, so entsteht ein Rotationskörper. Sein Volumen V berechnet man mit \[V = \pi \int_a^b(f(x))^2dx\]

\end{definition}

\begin{definition}

Eine Fläche A wird von den Graphen der Funktion der auf [a;b] differenzierbaren f und g über dem Interval [a;b] eingeschlossen. Rotiert diese Fläche A um die die x-Achse, berechnet man das Volumen V des  aufstehendes Körpers mit ($f(x) > g(x) auf [a;b]$)

\[V = \pi \int_a^b (f(x))^2dx - \pi \int_a^b (g(x))^2dx = \pi \int_a^b (f(x))^2 - (g(x))^2dx\]

\end{definition}


\end{document}