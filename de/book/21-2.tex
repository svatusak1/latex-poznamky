\documentclass{article}
\usepackage[shortlabels]{enumitem}
\usepackage{amsmath}
\usepackage{amssymb}
\usepackage{stmaryrd}

\newcommand{\mathcolorbox}[2]{\colorbox{#1}{$\displaystyle #2$}}

\newtheorem{definition}{Definition}

\begin{document}

\section{Eigenschaften der Intgrale}
\begin{definition}Eigenschaften von Integralen
\\
\begin{enumerate}[a)]
\item Linearität 
\begin{itemize}
\item \[\int_a^b c \cdot f(x) dx = c \cdot \int_a^b f(x) dx\]
\item \[\int_a^b g(x) + f(x) dx = \int_a^b g(x) dx + \int_a^b f(x) dx \]
\end{itemize}
\item Vertauschung der Grenzen\[ \int_a^b f(x) dx = -\int_b^a f(x) dx\]
\item Gleiche Grenzen \[\int_a^a f(x) dx = 0\]
\item Additivität \[\int_a^b f(x) dx + \int_b^c f(x) dx = \int_a^c f(x) dx\]
\item Monotonie \[f(x) \leq g(x) \Rightarrow \int_a^b f(x) dx \leq \int_1^b g(x) dx\]

\end{enumerate}

\subsection{Beweise von Eigenschaften den Integralen}
\subsubsection{Beweis von Linearität 1}

\[\text{Behauptung: } \int_a^b c \cdot f(x) dx = c \cdot \int_a^b f(x) dx\]
\[ \text{Beweis: } \int_a^b cf(x) dx = \int_a^b cF'(x) dx = [cF(x)]_a^b\]
\[ = cF(b) - cF(a) = c (F(b) - F(a)) = c \int_a^b f(x) dx \; \text{ q.e.d.}\]

\subsubsection{Beweis von Linearität 2}
\[\text{Behauptung: } \int_a^b g(x) + f(x) dx = \int_a^b g(x) dx + \int_a^b f(x) dx \]
\[ \text{Beweis: } \int_a^b g(x) + f(x) dx = [F(x) +G(x)]_a^b = F(b) +G(b) - F(b) - G(b) \]
\[= F(a) - F(b) + G(a) - G(b) = \int_a^b f(x) dx + \int_a^b g(x) dx \; \text{ q.e.d.}\]

\subsection{Anwendung der Eigenschaften von Integralen}

\begin{itemize}

\item 
\[\int_1^2 \frac {X^3-2}{x^2}dx = \int_1^2 \frac {X^3}{x^2}dx - \int_1^2 \frac 2{x^2}dx\]
\[ = \int_1^2 x \;dx - 2 \cdot \int_1^2 \frac 1 {x^2}dx\]

\item
 \[\int_2^4 \frac{\sqrt{x}+x^2}{x-1} dx - \int_2^4 \frac{\sqrt{x}+1}{x-1}dx = \int_2^4 \frac{x^2+1}{x-1}dx\]
\[= \int_2^4 \frac{(x+1)(x-1)}{x-1}dx = \int_2^4 x+1 dx\]
\end{itemize}



\end{definition}
\section{Flächeinhalt zwischen Schaubild und x-Achse}

\begin{definition}
Sei f eine stetige Funktion auf I = [a;b]. Bei der Berechnung des Flächeinhalts zwischen den Graphen $G_f$ einer Funktion f und den x-Achse über  denn Intervall I = [a;b] geht man wie folgt vor:

\begin{enumerate}[i)]

\item Bestimmung der Nullstelen ($f(x) \overset{!}{=}0$) von f auf I = [a;b]
\item Man berechnet die orientierten Flächeinhalt (Itegrale) über den Teilintervallen ($[a;x_0],[x_0;b]$)
\item Man addiert die Betrage der einzelnen Integralen (Inhalte der Teilflächen)

\end{enumerate}

\end{definition}

\begin{definition}
Sei f eine stetige Funnkion auf I = [a;b]. Bei der BErechnung des Flächeinhalts zwischen den Graphen $G_f$ einer Funktion f und der x-Achse geht man wie folgt vor:

\begin{enumerate}[i)]

\item Bestimmung der Nullstelen ($f(x) \overset{!}{=}0$) von f $\Rightarrow \; x_{01}, x_{02}, \dots$

\item Man berechnet die orientierten Flächeinhalt (Itegrale) über den Teilintervallen $[x_{01}, x_{02}]$ etc.

\item Man addiert die Betrage der einzelnen Integralen (Inhalte der Teilflächen)
\end{enumerate}


\end{definition}




\end{document}