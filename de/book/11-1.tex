\documentclass{article}
\usepackage[shortlabels]{enumitem}
\usepackage{amsmath}
\usepackage{amssymb}
\usepackage{stmaryrd}

\usepackage{pgfplots}
\newtheorem{definition}{Definition}

\title{Integralen}
\begin{document}

\section{Rekonstruieren einer größe}

\subsection{Beispiel1: Darstellung als Tabelle}

\begin{center}
\begin{tabular}{ | c || c | c | c | }
\hline
Intervall & [0;10] & [10;15] & [15;20] \\
\hline \hline
Flächeinhalt & 300FE & 0FE & 100FE\\
\hline
Wegänderung & 300m & 0m & -100m\\
\hline
orientierte Flächeinhalt  & 300FE & 0FE & -100FE\\
\hline
zurückgelegte Strecke & 300m &  0m & 100m\\
\hline
Ortspunkt von Start & 300m &  300m & 200m\\
\hline
\end{tabular}
\end{center}

\subsection{Beispiel2: Darstellung als Tabelle}

\begin{center}
\begin{tabular}{ | c || c | c | c | c | c | c | }

\hline
Intervall & [0;3] & [3;5] & [5;6] & [6;7] & [7;8] & [8;9] \\
\hline \hline
Flächeinhalt & 30FE & 40FE & 10FE & 0FE & 5FE & 10FE\\
\hline
Wegänderung & 30m & 40m & 10m & 0m & -5m & -10m\\
\hline
orientierte Flächeinhalt  & 30FE & 40FE & 10FE & 0FE & -5FE & -10FE\\
\hline
zurückgelegte Strecke & 30m & 40m & 10m & 0m & 5m & 10m\\
\hline
Ortspunkt von Start & 30m & 70m & 80m & 80m & 75m & 65m \\
\hline
\end{tabular}
\end{center}

\begin{definition}Ist der Graph einer momentans Änderungrate einer Größe gegeben, so gibt der \underline{Flächeinhalt zwischen dem Graphen un der x-Achse der Gesamt Änderung} der Größe an (unter der x-Achse: negative Änderung)
Mithilfe des Anfangswerts lassen sich \underline{die Werte der Gröse rekonstruieren}.
\end{definition}
Bemerkung:
Wird der Flächeinhalt zwischen einem Graphen und der x-Achse \underline{oberhalb der x-Achse positiv und unterhalb negativ gezält}, so spricht man von  einem orientierten Flächeinhalt.
\\
\\
Beispiele:
\begin{center}
\begin{tabular}{ c | c }
momentane Anderungsrate & Grose als Flacheinhalt \\
\hline \hline
Geschwindigkeit ( $\frac ms$) & Strecke (m) \\
Durchflussmenge ($\frac {m^3}s$) & Volumen ($m^3$) \\
Verkaufszahlen ($\frac 1s$) & absolute Verkaufszahl\\
Beschleunigung ($\frac m{s^2}$) & Geschwindigkeit ($\frac ms$)\\
Temperaturänderung ($\frac K {km}$) & Temperatur (Kelvin) \\
\end{tabular}
\end{center}


\end{document}