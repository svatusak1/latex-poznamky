\documentclass[twocolumn]{article}
\usepackage{mathtools}
\usepackage{amssymb}
\begin{document}

\section{35:6}
\subsection{gegeben: }
 $f(x) = 9-x^2 \text{für} x \in [0;3]$
Rechteck (0;0) P(u; f(u)) 
Rechteck solte die maximalem Umfang haben
\subsection{Prozess}
\[U = a*b \]
\[a = u\]
 \[b = f(u)\]
 \[U = u*f(u) = u*(9-u^2) = -u^3+9u\]
 \[U' = -3u^2+9\]
 \[U'' = -6u\]
 \[U' = 0 \rightarrow -3u^2+9 = 0\]
 \[3u^2 = 9\quad u^2 = 3\]
 \[u = \pm \sqrt3\]
 \[U''(-\sqrt3) > 0 \rightarrow Maximum\]
  \[U''(\sqrt3) < 0 \rightarrow Minimum\]
 \[D = \mathbb{R} \]
 \[\pm \sqrt3 \in D\]
 \subsection{Lösung}
 Recheck mit Punkt $P(\sqrt3; f(\sqrt3))$ hat der maximalen Umfang
 
 \section{35:7}
 \subsection{gegeben:}
 \begin{itemize}
 \item von Oben offene Kartonkiste
 \item quadratischer Grundfläche
 \item flächeinhalt = $100cm^2 $
 \item möglichst größte Volume.
 \end{itemize}
 
 \subsection{Prozess}
 \[A = a^2 + 4 ab = 100\]
 \[V= a^2 b\]
\[ b = \frac {100-a^2}{4a}\]
\[V = a^2 \cdot \frac {100-a^2}{4a} = -\frac{a^3}4+25a\]
\[V' = -\frac34 a^2+25 \]
\[V'' = -\frac32 a\]
\[V' = 0 \rightarrow -\frac34 a^2+25 = 0\]
\[\frac34 a^2 = 25\]
\[a^2 = \frac{100}3\]
\[a = \pm \sqrt{\frac{100}3}\]
\[V''(\dots) <0 \rightarrow Maximum\]
\[V''(-\dots) >0 \rightarrow Minimum\]
\[D = \mathbb{R}^0_+\]
\[\sqrt{\frac{100}3} \in D\]
\[V(0) = 0\]

\subsection{Lösung}
\[a = \sqrt{\frac{100}3} \approx 0,2\]
\[b = \frac {100-a^2}{4a} = \frac {100-(\sqrt{\frac{100}3})^2}{4\sqrt{\frac{100}3}} =\frac5{\sqrt3} \approx 2,9\]
\[V = (\sqrt{\frac{100}3})^2 \cdot \frac5{\sqrt3} = \frac {500}{3\sqrt3} \approx 96,2\]

\section{36:14}
\subsection{gegeben:}
\begin{itemize}
\item Tunnel $= f(x) : -3 < x <3$
\item $f(x) = -\frac12 x^2+4,5 $
\item Wagen: 2,4 x 3,5 m
\item Vermutung: Abstand zwischen rechten oben Punkt des Wagen kann mann mit $\sqrt{(1,2-u)^2+(3,5-(0,5u^2+4,5))^2}$ berechnen.
\item berechne u mit d(u) wird minimal und den Abstand
\end{itemize}

\subsection {Prozess}
Nullpunkte der Parabel(Tunnel): $-\frac12 x^2+4,5 = 0$
\[\frac12 x^2 = 4,5\]
\[x = \pm 3 \rightarrow \text{Mittelpunkt ist }0\]
$\Rightarrow $rechte oben Punkt der Wagen ist P(1,2 ; 3,5)
Von hier kann man die Pythagoras benutzen $\Rightarrow$ 
\[d^2 = a^2 + b^2\]
\[a = u-1,2\]
\[ b = 3,5-f(u) = 3,5-(0,5u^2+4,5))\]
\[ \Rightarrow d = \sqrt{(1,2-u)^2+(3,5-(0,5u^2+4,5))^2}\]

kleinste Abstand:
\[d =\sqrt{(1,2-u)^2+(3,5-(0,5u^2+4,5))^2} = \sqrt{(1,2-u)^2+(3,5 - 0,5u^2-4,5)^2} \]
\[d= \sqrt{(1,2-u)^2+(- 0,5u^2-1)^2} = \sqrt{1,44+u^2-2,4u+\frac14u^4-u^2+1}\]
\[d = \sqrt{\frac14u^4-2,4u+2,44}\]
\[d' = \frac12 (\frac14u^4-2,4u+2,44) ^{-\frac12} \cdot (u^3-2,4) = \frac{u^3-2,4}{2 \sqrt{(\frac14u^4-2,4u+2,44)}}\]
\[d' = 0 \Leftrightarrow u^3-2,4 = 0\]
\[u^3 = 2,4\]
\[u = \sqrt[3]{2,4} \approx 1,34\]
\[d = \sqrt{\frac14\sqrt[3]{2,4}^4-2,4\sqrt[3]{2,4}+2,44} \approx  0,5\]









\end{document}