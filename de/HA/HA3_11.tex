\documentclass{book}
\usepackage[shortlabels]{enumitem}
\usepackage{amsmath}
\usepackage{amssymb}
\usepackage{marvosym}
\let\marvosymLightning\Lightning

\usepackage{pgfplots}
\pgfplotsset{compat=newest}
\usepgfplotslibrary{fillbetween}

\newcommand{\mathcolorbox}[2]{\colorbox{#1}{$\displaystyle #2$}}

\newcommand*\circled[1]{\tikz[baseline=(char.base)]{
\node[shape=circle, draw, inner sep=2pt] (char) {#1};}}

\newtheorem{definition}{Definition}

\begin{document}

\begin{itemize}
    \item[8]
        \[f(x) = x^3 - 6x^2 +12x-7\]
        g(x) ist um 4 in x-Richtung und -3 in y-Richtung verschoben uns ist zum W = (2|1) punktsymetrisch
        \begin{enumerate}[a)]
            \item \[g(x) = (x-4)^3 - 6(x-4)^2 + 12(x-4)-7 -3\]
            \item $W_g = (6|-2)$ der Punkt ist ähnlich wie der Graph verschoben
        \end{enumerate}
    \item[10]
        \[f(x) = x\cdot e^x\]
        \begin{enumerate}[a)]
            \item \hfill \break
                \begin{enumerate}
                    \item[Tiefpunkt]
                        \[f'(x) = e^x + x \cdot e^x = e^x(x+1)\]
                        \[f'(x) = 0 = e^x(x+1)\]
                        \[\Rightarrow x = -1 \Rightarrow P_T = (-1|f(-1)) = (-1| - \frac{1}{e})\]
                    \item[Wendepunkt]
                        \[f''(x) = f'(x)' = e^x +e^x + x \cdot e^x = e^x(2+x)\]
                        \[f''(x) = 0 = e^x(2+x)\]
                        \[\Rightarrow x = -2 \Rightarrow P_W = (-2|f(-2)) = (-2|- \frac{2}{e^2} )\]
                \end{enumerate}
            \item $ g(x) = -(x+2) \cdot e^{x+2} +1 $
                \begin{itemize}
                    \item Spiegelung an der x-Achse
                    \item Verschiebung im positive x-Richtung um 2
                    \item Verschiebung im positive y-Richtung um 1
                \end{itemize}
                \begin{enumerate}
                    \item[Tiefpunkt]
                        $H(3|- \frac 1{e}-1)$
                    \item[Wendepunkt]
                        $H(1|- \frac{2}{e^2}-1 )$
                \end{enumerate}
        \end{enumerate}

    \item[11a)]
        \[ \int_0^1x^3dx = [\frac{1}{4}x^4]_0^1 = \frac{1}{4} \]
        \[ \int_2^3 ((x-2)^3+1)dx = \frac{1}{4} + 1\]
    \item[12]
        f ist punktsymetrisch zum Ursprung
        g ist achsensymetrisch zum y-Achse
        \begin{enumerate}
            \item[b)] $h(x) = 2g(x) +1$
                \[h(-x) = 2g(-x) +1 = 2g(x) +1 = h(x) \]
                $\Rightarrow$ y-Achse symetrisch
            \item[d)] $h(x) = x \cdot f(x)$
                \[h(-x) = -x \cdot f(-x) = -x (-f(x)) = xf(x) = h(x)\]
                $\Rightarrow$ y-Achse symetrisch
            \item[e)] $ h(x) = f(g(x))$
                \[h(-x) = f(g(-x)) = f(g(x)) =h(x)\]
                $\Rightarrow$ y-Achse symetrisch
        \end{enumerate}            
    \item[14]
        
\end{itemize}
\end{document}
