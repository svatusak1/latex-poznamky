\documentclass{book}
\usepackage[shortlabels]{enumitem}
\usepackage{amsmath}
\usepackage{amssymb}
\usepackage{marvosym}
\let\marvosymLightning\Lightning

\usepackage{pgfplots}
\pgfplotsset{compat=newest}
\usepgfplotslibrary{fillbetween}

\newcommand{\mathcolorbox}[2]{\colorbox{#1}{$\displaystyle #2$}}

\newcommand*\circled[1]{\tikz[baseline=(char.base)]{
\node[shape=circle, draw, inner sep=2pt] (char) {#1};}}

\newtheorem{definition}{Definition}

\begin{document}


\[f_h(x) = \frac{3h-3}{(x+2)^h} \; h > 1 \; ; \; I = [0;\infty]\]
\[A(z) = \int_0^z \frac{3h-3}{(x+2)^h} dx = [-\frac 1 {h-1}\frac{3h-3}{(x+2)^{h-1}}]_0^z \]
\[ - \frac 1 {-h+1} \frac {3h-3}{z+2}^{h-1} - (-\frac 1{h-1} \frac {3h-3}{2^{h-1}})\]
\[A_h = A_h(z) \lim_{z\to\infty}A_h(z)=\frac 1 {h-1} \frac {3h-3}{2^{h-1}} = 1 \]
\[\frac {3h-3}{h-1} = 2^{h-1}\]
\[\ln \frac{3h-3}{h-1} = (h-1)\ln(2)\]
\[\ln 3\frac{h-1}{h-1}=(h-1)\ln(2)\]
\[ln (3) = (h-1)\ln(2)\]
\[h = 1+ \frac {\ln3}{\ln2}\]


\underline{Skizzieren Sie die Graphen der folgenden Funktion:}
\begin{enumerate}[i)]
    \item $f(x) = (x-1)(x-4)$
    \item $g(x) = (x-1)(x-4)^2$
    \item $i(x) = (x-1)^2(x-4)^2$
    \item $j(x) = (x-1)(x-4)^4$

\end{enumerate}

\begin{tikzpicture}
    \begin{axis}[
        legend pos=outer north east,
        title=Example,
        axis lines = center,
        xlabel = $x$,
        ylabel = $y$,
        variable = t,
        trig format plots = rad,
        xtick distance = 1,
        ]

        \addplot [
            samples=70,
            color=blue,
            ]
            {(x-1)*(x-4)};
        \addlegendentry{$ y=(x-1)*(x-4)$}
    
    \end{axis}
\end{tikzpicture}


Formulieren sie eine Vermutung für den Graphen def Funktion f mit $ l(x) = (x-1)^k(x-4)^m$

\end{document}
